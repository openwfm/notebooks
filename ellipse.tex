\documentclass{article}%
\usepackage{amsmath}%
\setcounter{MaxMatrixCols}{30}%
\usepackage{amsfonts}%
\usepackage{amssymb}%
\usepackage{graphicx}
%TCIDATA{OutputFilter=latex2.dll}
%TCIDATA{Version=5.50.0.2960}
%TCIDATA{CSTFile=40 LaTeX article.cst}
%TCIDATA{Created=Thursday, December 02, 2021 19:07:13}
%TCIDATA{LastRevised=Monday, December 06, 2021 21:59:38}
%TCIDATA{<META NAME="GraphicsSave" CONTENT="32">}
%TCIDATA{<META NAME="SaveForMode" CONTENT="1">}
%TCIDATA{BibliographyScheme=Manual}
%TCIDATA{<META NAME="DocumentShell" CONTENT="Standard LaTeX\Blank - Standard LaTeX Article">}
%BeginMSIPreambleData
\providecommand{\U}[1]{\protect\rule{.1in}{.1in}}
%EndMSIPreambleData
\newtheorem{theorem}{Theorem}
\newtheorem{acknowledgement}[theorem]{Acknowledgement}
\newtheorem{algorithm}[theorem]{Algorithm}
\newtheorem{axiom}[theorem]{Axiom}
\newtheorem{case}[theorem]{Case}
\newtheorem{claim}[theorem]{Claim}
\newtheorem{conclusion}[theorem]{Conclusion}
\newtheorem{condition}[theorem]{Condition}
\newtheorem{conjecture}[theorem]{Conjecture}
\newtheorem{corollary}[theorem]{Corollary}
\newtheorem{criterion}[theorem]{Criterion}
\newtheorem{definition}[theorem]{Definition}
\newtheorem{example}[theorem]{Example}
\newtheorem{exercise}[theorem]{Exercise}
\newtheorem{lemma}[theorem]{Lemma}
\newtheorem{notation}[theorem]{Notation}
\newtheorem{problem}[theorem]{Problem}
\newtheorem{proposition}[theorem]{Proposition}
\newtheorem{remark}[theorem]{Remark}
\newtheorem{solution}[theorem]{Solution}
\newtheorem{summary}[theorem]{Summary}
\newenvironment{proof}[1][Proof]{\noindent\textbf{#1.} }{\ \rule{0.5em}{0.5em}}
\begin{document}
Write the equation of an ellipse with horizontal axis $a$ and vertical axis
$b$ in parametric form%
\[
\left[
\begin{array}
[c]{c}%
x\\
y
\end{array}
\right]  =\left[
\begin{array}
[c]{c}%
a\cos s\\
b\sin s
\end{array}
\right]  ..
\]
Notable points:%
\[%
\begin{tabular}
[c]{ccc}%
$s$ & $\left(  x,y\right)  $ & vertex\\
$-\frac{1}{2}$ & $\left(  0,-b\right)  $ & bottom\\
$0$ & $\left(  a,0\right)  $ & right\\
$\frac{\pi}{2}$ & $\left(  0,b\right)  $ & top\\
$\pi$ & $\left(  -a,0\right)  $ & left
\end{tabular}
\]
Rotate by angle $\theta\in(-\pi,\pi]$ clockwise:%
\[
\left[
\begin{array}
[c]{c}%
x\\
y
\end{array}
\right]  =\left[
\begin{array}
[c]{cc}%
\cos\theta & \sin\theta\\
-\sin\theta & \cos\theta
\end{array}
\right]  \left[
\begin{array}
[c]{c}%
a\cos s\\
b\sin s
\end{array}
\right]  ,\quad s\in(-\pi,\pi]
\]
Multiplying out we get%
\[
\left[
\begin{array}
[c]{c}%
x\\
y
\end{array}
\right]  =\left[
\begin{array}
[c]{c}%
a\cos\theta\cos s+b\sin\theta\sin s\\
-a\sin\theta\cos s+b\cos\theta\sin s
\end{array}
\right]  ,\quad s\in(-\pi,\pi]
\]
and move the center vertically so that the point distance $c$ from the bottom
vertex on the $b$ axis is at $y=0$,%

\[
\left[
\begin{array}
[c]{c}%
x\\
y
\end{array}
\right]  =\left[
\begin{array}
[c]{c}%
a\cos\theta\cos s+b\sin\theta\sin s\\
-a\sin\theta\cos s+b\cos\theta\sin s+(b-c)\cos\theta
\end{array}
\right]
\]
This is the equation of the ellipse from the figure, and the rate of spread in
the direction normal to the fireline is%
\[
R=\max_{s}-a\sin\theta\cos s+b\cos\theta\sin s+(b-c)\cos\theta
\]
The find the highest point, set%
\[
y^{\prime}\left(  s\right)  =\frac{\partial}{\partial s}\left(  -a\sin
\theta\cos s+b\cos\theta\sin s+(b-c)\cos\theta\right)  =0
\]
which gives%
\[
a\sin\theta\sin s+b\cos\theta\cos s=0
\]
We can either divide by $\sin\theta\neq0$,
\[
\frac{\sin s}{\cos s}+\frac{b}{a}\frac{\cos\theta}{\sin\theta}=0,
\]
and compute $s$ from%
\[
s=-\arctan\left(  \frac{b\cos\theta}{a\sin\theta}\right)
\]
Using the arctan2 function in numpy%
\[
s=-\text{arctan2}\left(  b\cos\theta,a\sin\theta\right)
\]
correct result even for $\sin\theta=0.$ In any case, we get two solutions, $s$
and $s+\pi$,  substitute in the equation of the ellipse%
\[
y=-a\sin\theta\cos s+b\cos\theta\sin s+\left(  b-c\right)  \cos\theta
\]
and take the larger value:%
\[
R=\max\left\{  u,-u\right\}  +c\cos\theta,\quad u=-a\sin\theta\cos
s+b\cos\theta\sin s
\]





\end{document}